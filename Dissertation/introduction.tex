\chapter*{Введение}                         % Заголовок
\addcontentsline{toc}{chapter}{Введение}    % Добавляем его в оглавление

Среди координационных соединений особое внимание  привлекают $\beta-$дикетонаты дифторида бора, обладающие
интенсивной люминесценцией  \cite{butler2016mechanochromic,fedorenko2017new,zhang2018reversible,
guieu2015synthesis,wang2015red,mamiya2016photochemically}, способностью к формированию эксимеров 
\cite{mirochnik2003crystal,bukvetskii2010crystal} и эксиплексов \cite{chow1995exciplexes,chow1995exciplex},
люминесцентным термохромизмом \cite{mirocnik2005obratimyj} и жидкокристаллическим полиморфизмом 
\cite{turanova2010liquid,mayoral2011alkoxy,sanchez2012liquid,giziroglu20141}. $\beta-$Дикетонаты дифторида бора
находят применение в качестве лазерных красителей \cite{czerney1990german}, органических светодиодов 
\cite{derosa2015tailoring,daly2016blue} , оптических хемосенсоров \cite{zhai2016nanofibers,fraser2018oxygen,
samonina2015luminescent}, активных компонентов солнечных коллекторов \cite{halik2003german}, материалов для 
нелинейной оптики \cite{kammler1996second} и полимерных оптических материалов\cite{tanaka2013facile,
zhang2009role}.
Установление взаимосвязей электронного строения и спектральных характеристик $\beta-$дикетонатов дифторида бора 
открывает возможности для создания новых оптических материалов на их основе. Поэтому моделирование электронной 
структуры и интерпретация фотоэлектронных и абсорбционных спектров $\beta-$дикетонатов дифторида бора является 
актуальной задачей.

Люминесцентные свойства $\beta-$дикетонатов дифторида бора определяются природой заместителей в $\beta-$положениях
(\ref{RIS0}). Замещение в $\gamma-$положении перспективно для формирования комплексов, обладающих химическими 
свойствами необходимыми для иммобилизации на поверхности или внедрения в другие структуры. Кроме того, введение
$\gamma-$заместителя должно препятствовать свободному вращению ароматических групп в $\beta-$положениях, что 
может привести к батохромному сдвигу спектра люминесценции и увеличению квантового выхода.
В настоящий момент практически отсутствуют сведения о влиянии заместителей в $\gamma-$положении на оптические 
свойства $\beta-$дикетонатов дифторида бора \cite{svistunova2008alpha,tikhonov2018electronic}.
Поэтому заслуживает внимания изучение электронного строения $\gamma-$замещённых ацетилацетонатов дифторида бора 
$-$ простейших $\beta-$дикетонатов дифторида бора с функциональными группами в $\gamma-$положении.
Наиболее достоверную информацию об электронном строении химических соединений можно получить при совместном 
применении методов фотоэлектронной спектроскопии и квантовой химии \cite{nefedov1987elektronnaa,
nefedov1989elektronnaa,osmushko2016application}.
Применение метода абсорбционной спектроскопии и результатов моделирования возбуждённых состояний позволяет
устанавливать взаимосвязи молекулярной структуры и оптических свойств хелатов бора на основе анализа электронных
эффектов замещения \cite{karasev2006fotofizika,kazachek2015excited,margar2016fluorescent},а также давать
рекомендации для синтеза новых комплексов с заданными положениями максимумов спектра флюоресценции
\cite{ponce2018searching}.\\


\newcommand{\actuality}{}
\newcommand{\progress}{}
\newcommand{\aim}{{\textbf\aimTXT}}
\newcommand{\tasks}{\textbf{\tasksTXT}}
\newcommand{\novelty}{\textbf{\noveltyTXT}}
\newcommand{\influence}{\textbf{\influenceTXT}}
\newcommand{\methods}{\textbf{\methodsTXT}}
\newcommand{\defpositions}{\textbf{\defpositionsTXT}}
\newcommand{\reliability}{\textbf{\reliabilityTXT}}
\newcommand{\probation}{\textbf{\probationTXT}}
\newcommand{\contribution}{\textbf{\contributionTXT}}
\newcommand{\publications}{\textbf{\publicationsTXT}}

\input{common/characteristic} % Характеристика работы по структуре во введении и в автореферате не отличается (ГОСТ Р 7.0.11, пункты 5.3.1 и 9.2.1), потому её загружаем из одного и того же внешнего файла, предварительно задав форму выделения некоторым параметрам

\textbf{Объем и структура работы.} Диссертация состоит из~введения,
\formbytotal{totalchapter}{глав}{ы}{}{},
заключения и
\formbytotal{totalappendix}{приложен}{ия}{ий}{}.
%% на случай ошибок оставляю исходный кусок на месте, закомментированным
%Полный объём диссертации составляет  \ref*{TotPages}~страницу
%с~\totalfigures{}~рисунками и~\totaltables{}~таблицами. Список литературы
%содержит \total{citenum}~наименований.
%
Полный объём диссертации составляет
\formbytotal{TotPages}{страниц}{у}{ы}{}, включая
\formbytotal{totalcount@figure}{рисун}{ок}{ка}{ков} и
\formbytotal{totalcount@table}{таблиц}{у}{ы}{}.
Список литературы содержит
\formbytotal{citenum}{наименован}{ие}{ия}{ий}.
